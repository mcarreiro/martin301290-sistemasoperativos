\section{Ejercicio 2}

%SI QUIEREN AGREGAR IMAGENES COPIEN EL SIGUIENTE CODIGO
%\begin {center}
%\includegraphics[width=12cm]{./graphEj1.jpg}
% grafico.eps: 0x0 pixel, 300dpi, 0.00x0.00 cm, bb=50 50 410 302
%\end {center}
\begin {center}
\includegraphics[width=16cm]{../simusched/outputs/outEj2.png}
\end {center}


En el diagrama se puede observar como el scheduler ejecuta las tareas en el orden en que entraron. Esto es debido al tipo de scheduler que utilizamos para correr el lote de tareas, este scheduler (FIFO / FCFS) además de ejecutar las tareas en el orden de entrada, no cambia de tarea hasta que la que está ejecutando termine. 

Por esto se puede observar que la primera tarea realiza 30 ciclos de reloj y recién después pasa a la próxima tarea. Esto es porque la primera tarea que se ejecuta es "TaskCPU 30" la cual realiza exactamente 30 ciclos de reloj. Lo mismo sucede con las otras 2 tareas, se ejecutan por completo y no hace el switch hasta finalizar la tarea acutal.

También en el gráfico se puede observar que el tiempo de bloqueo no es siempre el mismo pero que siempre se encuentra dentro de los rangos enviados (establecidos en el lote de tareas).

Finalmente vale destacar que las tareas interactivas cumplen lo pedido ya que la primera tarea ejecuta 2 bloqueos de aproximadenete 10 ciclos y la segunda 5 bloqueos de aproximadamente 30 ciclos cada uno. Lo que se ajusta los parametros de entrada ya que la primera tarea tiene un rango de duracion por ciclo de entre 1 y 10 y la segunda entre 20 y 50.
