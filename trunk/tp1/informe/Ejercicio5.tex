\section{Ejercicio 5}
Se indican las simplificaciones tomadas:\\
Primero queremos aclarar que se decicidió que a cada tarea se le asigna 1 ticket y este número irá 
incrementándose a medida que gane compensaciones por bloqueo. A partir de este momento, la tarea tiene más probabilidad de salir ganador en los próximos sorteos. Cuando resulta ganador de la lotería, esta compensación se le es retirada volviendo a 1, cantidad inicial.\\
Decidimos omitir el hecho de que una tarea pueda subdividirse en threads, junto con el manejo
de tickets del usuario. Con esto queremos decir que al no existir usuarios (solo uno) no existe la posibilidad del cambio
de cantidad de tickets bajo un mismo usuario.

%SI QUIEREN AGREGAR IMAGENES COPIEN EL SIGUIENTE CODIGO
%\begin {center}
%\includegraphics[width=12cm]{./graphEj1.jpg}
% grafico.eps: 0x0 pixel, 300dpi, 0.00x0.00 cm, bb=50 50 410 302
%\end {center}
