\section{Ejericio 2}
\subsection{Descripción}
Para el ejercicio 2 lo que hicimos fue basarnos en el backend$\_$mono. En el main, cuando la ejecución se queda  \textit{loopeando} en el while(true) le sacamos la línea que cierra la conexión (en el momento que esta se etablece exitosamente) e hicimos que al cliente lo atienda un thread a través de la función \textbf{wrapper} (que fué necesario crear para que reciba un puntero a void, lo castee a int y se lo envíe a la funcion ya existente \textbf{atendedor$\_$de$\_$jugador}). De esta manera cada conexión entrante es atendida por un thread y se aceptan multiples conexiones. Luego como cada thread escribe sobre los mismos tableros (el de palabras y el de letras) fue necesario crear semáforos (utilizando la librería creada en el ejercicio anterior). Un semáforo para el tablero de palabras que hace que no puedan escribir 2 threads distintos al mismo tiempo sobre este, osea que no puedan gritar  \textbf{palabra} exactamente al mismo tiempo y un tablero de semáforos para cada posicion del tablero de letras. Este último es para que a medida que van formando las palabras no puedan escribir una letra sobre el mismo casillero al mismo tiempo. \\
Finalmente agregamos estos semáforos a la hora de escribir/leer alguno de los dos tableros. Esto podemos observarlo mejor en los siguientes pseudocódigos(vamos a poner el pseudocódigo de lo que agregamos, no de lo que ya estaba):\\
\\

\subsection{Pseudocódigo}
\mbox{}\\
\textbf {main}
\begin{codebox}
\li (...)
\li Inicializamos los semáforos
\li (...)
\li Cuando llega una nueva conexión 
\li \textbf{Si} se establece exitosamente \Do
\li	Creamos un thread
\li	hacemos que este nuevo cliente sea atendido por el thread a través de la función wrapper \End
\end{codebox}


\mbox{}\\
\textbf {wrapper}
\begin{codebox}
\li Casteo el parámetro de entrada a int y se lo envío a atendedor$\_$de$\_$jugador
\end{codebox}

\mbox{}\\
\textbf{atendedor$\_$de$\_$jugador}
\begin{codebox}
\li (...)
\li \textbf{Si} estoy agregando una letra\Do
\li	pido el recurso para escribir esa casilla
\li	escribo la letra
\li	libero el recurso de la casilla \End
\li (...)
\li \textbf{Si} quiero agregar una palabra\Do
\li	pido el recurso para escribir el tablero de palabras
\li	escribo todas las letras
\li	libero el recurso para el tablero de palabras \End
\end{codebox}

\mbox{}\\
\textbf{enviar$\_$tablero}
\begin{codebox}
\li pido el recurso para leer el tablero de palabras
\li leo todo el tablero
\li libero el recurso para el tablero de palabras
\end{codebox}

\mbox{}\\
\textbf{quitar$\_$letras} 
\begin{codebox}
\li \textbf{Para} cada casilla del tablero de letras \Do
\li	pido el recurso para escribir esa casilla
\li	escribo el valor VACÍO
\li	libero el recurso \End
\end{codebox}
\mbox{}\\
\textbf{es$\_$ficha$\_$valida$\_$en$\_$palabra}
\begin{codebox}
\li (...)
\li Pido el recurso para leer la casilla
\li \textbf{Si} la casilla esta vacia\Do
\li	libero el recurso
\li	devuelvo false\End
\li libero el recurso
\li (...)
\li \textbf{Si} no es la primera letra de la palabra \Do
\li	(...)
\li	\textbf{Para} cada casilla entre la nueva letra y las ya puestas \Do
\li		Pido recurso para leer esa casilla
\li		\textbf{Si} está vacía \Do
\li			libero el recurso
\li			devuelvo false \End
\li		libero el recurso \End \End
\li						
\li (...)
\end{codebox}

