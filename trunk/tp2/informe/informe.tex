\documentclass[8pt, A4]{article}

%Margenes de la pagina.  otra opcion, usar \usepackage{a4wide}
\usepackage[paper=a4paper, left=0.8cm, right=0.8cm, bottom=1.3cm, top=0.9cm]{geometry}
\usepackage{color}

%este paquete permite incluir acentos.  Notar que espera un formato ANSI-blah de archivo.  Si en lugar de eso se tiene un utf8 (usual en los linux), entonces usar \usepackage[utf8]{inputenc}
\usepackage[utf8]{inputenc}

%Este paquete es para que algunos titulos (como Tabla de Contenidos) esten en castellano
\usepackage[spanish]{babel}

%El siguiente paquete permite escribir la caratula facilmente
\usepackage{caratula}

\usepackage{aed2-symb,aed2-itef,aed2-tad,aed2-tokenizer,modulos_diseno, ./algorithms/clrscode3e}
\usepackage{framed}
\usepackage{amsmath}

\usepackage{graphicx}

%Datos para la caratula
\materia{Sistemas Operativos}

\titulo{Trabajo Pr\'actico 2 - Scrable}


\integrante{Ortiz de Zarate, Juan Manuel}{403/10}{jmanuoz@gmail.com}
\integrante{Kujawski, Kevin}{459/10}{kevinkuja@gmail.com}
\integrante{Carreiro, Martin}{45/10}{carreiromartin@gmail.com}

\begin{document}
%numero de grupo
%{\hfill\Huge 10}
%esto construye la caratula
\maketitle 

 
% \tableofcontents

 \newpage

\section{Ejericio 1}
\subsection{Librería RWLock sin inanición}
Para implementar una librer\'ia \textbf{Read Write Lock libre de inanici\'on}, utilizamos semáforos POSIX mediante la librería de C \textit{semaphore.h}. La librería creada por nosotros consta de los mismos metódos que la clase RWLock brindada por la cátedra. Disponemos de \textbf{4} atributos privados: \textbf{readersQuantity} (contador de lectores, inicializado en 0), \textbf{mutexReaders} (para implementar la sección crítica a la hora de tocar el atributo readersQuantity), \textbf{writers} (Semáforo para saber si hay un escritor esperando, inicializado en 1) y \textbf{freeResource} (para saber si el recurso esta disponible, inicializado en 1).

\subsection{Pseudoc\'odigo}

A continuación el pseudocódigo de la nueva librería RWSemaphoreLock:\\
\mbox{}\\
\textbf WLOCK

Descripción:\\
En el wlock, primero hay que esperar que no haya lectores y luego  que el recurso no esté tomado por otros para poder obtenerlo.
\begin{codebox}

\li	writers.wait() 		\RComment Espero que no haya ningún escritor
\li	freeResource.wait() 	\RComment Espero que se libere el recurso y lo tomo 
\end{codebox}
\mbox{}\\\mbox{}\\
\textbf WUNLOCK
			
Descripción:\\
Primero se libera el recurso, indicando que el escritor se fue y puedan entrar los siguientes lectores o escritores.
Segundo se libera la exclusividad del escritor, realizando un signal de \textit{writers}. 
\begin{codebox}
\li	freeResource.signal()		\RComment Libero el recurso	
\li	writers.signal()		\RComment Libero la exclusividad de los escritores
\end{codebox}
\mbox{}\\\mbox{}\\
\textbf RLOCK

Descripción:\\
El algoritmo primero realiza un wait y signal del semáforo \textit{writers} para evitar la inanición de los lectores, ya que en caso de que haya un escritor antes se quedará esperando.
Luego, aumenta el número de lectores, y en caso de ser el primero toma el recurso, en otro caso no necesitara tomarlo ya que algún lector lo tomó antes. 
\begin{codebox}
\li	writers.wait()			\RComment Espero que no haya escritores, para no "colarme" y evitar la inanición.
\li	writers.post()			\RComment Libero el semaforo para que los que vengan puedan seguir.
\li	mutexReaders.wait()		\RComment Tomo el mutex
\li	readersQuantitiy++		\RComment Incremento la cantidad de lectores
\li	\textbf{Si} readersQuantity == 1 \Do	\RComment Si soy el primero... 
\li		freeResource.wait()\End	\RComment .. tomo el recurso para los lectores
\li	mutexReaders.signal()		\RComment Libero el mutex
\end{codebox}
\mbox{}\\\mbox{}\\
\textbf RUNLOCK

Descripción:\\
Para desbloquear a los lectores, primero se decrementa la cantidad de lectores y luego, en caso de ser el último, se libera el recurso. Todo dentro del mutex para asegurar exclusión mutua de la variable.
\begin{codebox}

\li	mutexReaders.wait() \RComment Espero el mutex para asegurarme el uso exclusivo sobre lectores

\li	readersQuantitiy- - \RComment Disminuyo la cantidad de lectores

\li	\textbf{Si} readersQuantitiy == 0 \Do \RComment Si soy el ultimo lector que quedaba ... 
\li	freeResource.signal()\End  \RComment ... libero el recurso así puede tomarlo un escritor (en el caso que hubiese uno esperando)

\li	mutexReaders.signal() \RComment Libero el mutex
\end{codebox}

  \newpage
Para el ejercicio 2 lo que hicimos fue basarnos en el backend_mono. En el main, cuando la ejecución se queda loopeando en el while(true) le sacamos la linea que cierra la conexión (en el momento que esta se etablece exitosamente) e hicimos que al cliente lo atienda un thread a través de la función 'wrapper' (que fué necesario crear para que reciba un puntero a void, lo castee a int y se lo envíe a la funcion ya existente 'atendedor_de_jugador'). De esta manera cada conexión entrante es atendida por un trhead y se aceptan multiples conexiones. Luego como cada thread escribe sobre los mismos tableros (el de palabras y el de letras) fue necesario crear semáforos (utilizando la librería creada en el ejercicio anterior). Un semáforo para el tablero de palabras que hace que no puedan escribir 2 threads distintos al mismo tiempo sobre este, osea que no puedan gritar 'palabra' exactamente al mismo tiempo y un tablero de semáforos para cada posicion del tablero de letras. Este último es para que a medida que van formando las palabras no puedan escribir una letra sobre el mismo casillero al mismo tiempo. \\
Finalmente agregamos estos semáforos a la hora de escribir/leer alguno de los dos tableros. Esto podemos observarlo mejor en los siguientes pseudocodigos(vamos a poner el pseudocódigo de lo que agregamos, no de lo que ya estaba):\\
\\


MAIN \\
...
Inicializamos los semaforos\\
...
Cuando llega una nueva conexion \\
Si se establece exitosamente\\
	Creamos un thread\\
	hacemos que este nuevo cliente sea atendido por el thread a traves de la funcion wrapper\\


\\

wrapper\\
Casteo el parametro de entrada a int y se lo envio a atendedor_de_jugador


atendedor_de_jugador\\ \\
...
Si estoy agregando una letra\\
	pido el recurso para escribir esa casilla
	escribo la letra
	libero el recurso de la casilla
...
Si quiero agregar una palabra
	pido el recurso para escribir el tablero de palabras
	escribo todas las letras
	libero el recurso para el tablero de palabras


enviar_tablero\\

pido el recurso para leer el tablero de palabras
leo todo el tablero
libero el recurso para el tablero de palabras


quitar_letras\\

para cada casilla del tablero de letras
	pido el recurso para escribir esa casilla
		escribo el valor VACIO
	libero el recurso

es_ficha_valida_en_palabra\\
...
Pido el recurso para leer la casilla
si la casilla esta vacia
	libero el recurso
	devuelvo false
libero el recurso
...
si no es la primera letra de la palabra 
	...
	para cada casilla entre la nueva letra y las ya puestas
		pido recurso para leer esa casilla
		si esta vacia
			libero el recurso
			devuelvo false
		libero el recurso
						
...





\end{document}
