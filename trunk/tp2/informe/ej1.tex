\section{Ejericio 1}
\subsection{Librería RWLock sin inanición}
Para implementar una librer\'ia \textbf{Read Write Lock libre de inanici\'on}, utilizamos semáforos POSIX mediante la librería de C \textit{semaphore.h}. La librería creada por nosotros consta de los mismos metódos que la clase RWLock brindada por la cátedra. Disponemos de \textbf{4} atributos privados: \textbf{readersQuantity} (contador de lectores, inicializado en 0), \textbf{mutexReaders} (para implementar la sección crítica a la hora de tocar el atributo readersQuantity), \textbf{writers} (Semáforo para saber si hay un escritor esperando, inicializado en 1) y \textbf{freeResource} (para saber si el recurso esta disponible, inicializado en 1).

\subsection{Pseudoc\'odigo}

A continuación el pseudocódigo de la nueva librería RWSemaphoreLock:\\
\mbox{}\\
\textbf WLOCK

Descripción:\\
En el wlock, primero hay que esperar que no haya lectores y luego  que el recurso no esté tomado por otros para poder obtenerlo.
\begin{codebox}

\li	writers.wait() 		\RComment Espero que no haya ningún escritor
\li	freeResource.wait() 	\RComment Espero que se libere el recurso y lo tomo 
\end{codebox}
\mbox{}\\\mbox{}\\
\textbf WUNLOCK
			
Descripción:\\
Primero se libera el recurso, indicando que el escritor se fue y puedan entrar los siguientes lectores o escritores.
Segundo se libera la exclusividad del escritor, realizando un signal de \textit{writers}. 
\begin{codebox}
\li	freeResource.signal()		\RComment Libero el recurso	
\li	writers.signal()		\RComment Libero la exclusividad de los escritores
\end{codebox}
\mbox{}\\\mbox{}\\
\textbf RLOCK

Descripción:\\
El algoritmo primero realiza un wait y signal del semáforo \textit{writers} para evitar la inanición de los lectores, ya que en caso de que haya un escritor antes se quedará esperando.
Luego, aumenta el número de lectores, y en caso de ser el primero toma el recurso, en otro caso no necesitara tomarlo ya que algún lector lo tomó antes. 
\begin{codebox}
\li	writers.wait()			\RComment Espero que no haya escritores, para no "colarme" y evitar la inanición.
\li	writers.post()			\RComment Libero el semaforo para que los que vengan puedan seguir.
\li	mutexReaders.wait()		\RComment Tomo el mutex
\li	readersQuantitiy++		\RComment Incremento la cantidad de lectores
\li	\textbf{Si} readersQuantity == 1 \Do	\RComment Si soy el primero... 
\li		freeResource.wait()\End	\RComment .. tomo el recurso para los lectores
\li	mutexReaders.signal()		\RComment Libero el mutex
\end{codebox}
\mbox{}\\\mbox{}\\
\textbf RUNLOCK

Descripción:\\
Para desbloquear a los lectores, primero se decrementa la cantidad de lectores y luego, en caso de ser el último, se libera el recurso. Todo dentro del mutex para asegurar exclusión mutua de la variable.
\begin{codebox}

\li	mutexReaders.wait() \RComment Espero el mutex para asegurarme el uso exclusivo sobre lectores

\li	readersQuantitiy- - \RComment Disminuyo la cantidad de lectores

\li	\textbf{Si} readersQuantitiy == 0 \Do \RComment Si soy el ultimo lector que quedaba ... 
\li	freeResource.signal()\End  \RComment ... libero el recurso así puede tomarlo un escritor (en el caso que hubiese uno esperando)

\li	mutexReaders.signal() \RComment Libero el mutex
\end{codebox}
