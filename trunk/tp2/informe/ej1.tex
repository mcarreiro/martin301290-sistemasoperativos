Para implementar una librería Read_Write Lock libre de inanición, utilizamos semáforos POSIX mediante la librería de C semaphore.h. La librería creada por nosotros consta de los mismo metódos que la clase RWLock birndada por la cátedra. Disponemos de 4 atributos privados: readersQuantity (contador de lectores, inicializado en 0), mutexReaders (para implementar la sección crítica a la hora de tocar el atributo readersQuantity), writers (Semáforo para saber si hay un escritor esperando, inicializado en 1) y freeResource (para saber si el recurso esta disponible, inicializado en 1).

A continuación el pseudocodigo de la nueva libreria RWSemaphoreLock:

WLOCK
//espero que no haya ningún escritor
hayEscritores.wait()
//espero que se libere el recurso y lo tomo
recurso.wait()

WUNLOCK
-------				
	freeResource.signal()		// Libero el recurso	
	writers.signal()		// Libero la exclusividad de los escritores

RLOCK
-----
	writers.wait()			// Espero que no haya escritores, para no "colarme" y evitar la inanición.
	writers.post()			// Libero el semaforo para que los que vengan puedan seguir.
	mutexReaders.wait()		// Tomo el mutex
	readersQuantitiy++		// Incremento la cantidad de lectores
	if(readersQuantity == 1)	// Si soy el primero... 
		freeResource.wait()	// .. Tomo el recurso para los lectores
	mutexReaders.signal()		// Libero el mutex

RUNLOCK
//espero el mutex para asegurarme el uso exclusivo sobre lectores
mutexLectores.wai().
//disminuyo la cantidad de lectores
lectores--
//si soy el ultimo lector que quedaba libero el recurso así puede tomarlo un escritor (en el caso que hubiese uno esperando)
if(lectores == 0)
	recurso.signal()
//libero el mutex
mutexLectores.signal()
