\section{Ejericio 1}
\subsection{Librería RWLock sin inanición}
Para implementar una librer\'ia Read Write Lock libre de inanici\'on, utilizamos semáforos POSIX mediante la librería de C semaphore.h. La librería creada por nosotros consta de los mismo metódos que la clase RWLock birndada por la cátedra. Disponemos de 4 atributos privados: readersQuantity (contador de lectores, inicializado en 0), mutexReaders (para implementar la sección crítica a la hora de tocar el atributo readersQuantity), writers (Semáforo para saber si hay un escritor esperando, inicializado en 1) y freeResource (para saber si el recurso esta disponible, inicializado en 1).\\
\\
A continuación el pseudocodigo de la nueva libreria RWSemaphoreLock:\\
\\
\subsection{Pseudoc\'odigo}

\begin{codebox}
WLOCK\\
-------\\
Descripción:\\
En el wlock, primero hay que esperar que no haya lectores, y luego, hay que esperar que el recurso no esté tomado por otros para poder obtenerlo.\\

/**espero que no haya ningún escritor**/ \\
hayEscritores.wait() \\
/**espero que se libere el recurso y lo tomo**/ \\
recurso.wait() \\
\end{codebox}
WUNLOCK\\
-------		\\		
Descripción:\\
Primero se libera el recurso, indicando que el escritor se fue y puedan entrar los siguientes lectores o escritores.
Segundo se libera la exclusividad del escritor. \\
	freeResource.signal()		// Libero el recurso	\\
	writers.signal()		// Libero la exclusividad de los escritores\\

RLOCK\\
-----\\
Descripción:\\
El algoritmo primero realiza un wait y signal del semaforo writers para evitar la inanición de los lectores, ya que en caso de que haya un escritor antes se quedará esperando.
Luego, aumenta el numero de lectores, y en caso de ser el primero toma el recurso, en otro caso no necesitara tomarlo ya que algún lector lo tomó antes. \\

	writers.wait()			// Espero que no haya escritores, para no "colarme" y evitar la inanición.\\
	writers.post()			// Libero el semaforo para que los que vengan puedan seguir.\\
	mutexReaders.wait()		// Tomo el mutex\\
	readersQuantitiy++		// Incremento la cantidad de lectores\\
	if(readersQuantity == 1)	// Si soy el primero... \\
		freeResource.wait()	// .. Tomo el recurso para los lectores\\
	mutexReaders.signal()		// Libero el mutex\\
\\
RUNLOCK\\
--------\\
Descripción:\\
Para desbloquear a los lectores, primero se decrementa la cantidad de lectores y luego, en caso de ser el último, se libera el recurso. Todo dentro del mutex para asegurar exclusión mutua de la variable.\\
//espero el mutex para asegurarme el uso exclusivo sobre lectores\\
mutexLectores.wai().\\
//disminuyo la cantidad de lectores\\
lectores--\\
//si soy el ultimo lector que quedaba libero el recurso así puede tomarlo un escritor (en el caso que hubiese uno esperando)\\
if(lectores == 0)\\
	recurso.signal()\\
//libero el mutex\\
mutexLectores.signal()\\
