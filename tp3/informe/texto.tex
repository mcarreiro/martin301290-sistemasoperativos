\section{Enunciado}

El problema consistía en modificar el servidor dado por la cátedra, de forma tal que varios servidores puedan proveer un recurso compartido a sus clientes
sin deadlock, inanición, livelock.\\
El algoritmo para realizar esta sincronización es una implentación del algoritmo de Ricart y Agrawala.

\section{Solución}

Para realizar dicha tarea agregaoms los handlers necesarios para cumplir con el protocolo servidor-cliente y entre servidores.
A su vez, tal como describe el paper, para mantener la sincronización y evitar que siempre un servidor tenga prioridad sobre los otros, agregamos
un número de secuencia para desempatar en caso de pedidos simultáneos.\\
La correcta implementación del algoritmo nos garantiza una solución libre de deadlock, livelock,
inanición o violación de la exclusión mutua.

\section{Tests}
