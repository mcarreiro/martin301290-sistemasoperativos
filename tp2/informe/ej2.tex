Para el ejercicio 2 lo que hicimos fue basarnos en el backend_mono. En el main, cuando la ejecución se queda loopeando en el while(true) le sacamos la linea que cierra la conexión (en el momento que esta se etablece exitosamente) e hicimos que al cliente lo atienda un thread a través de la función 'wrapper' (que fué necesario crear para que reciba un puntero a void, lo castee a int y se lo envíe a la funcion ya existente 'atendedor_de_jugador'). De esta manera cada conexión entrante es atendida por un trhead y se aceptan multiples conexiones. Luego como cada thread escribe sobre los mismos tableros (el de palabras y el de letras) fue necesario crear semáforos (utilizando la librería creada en el ejercicio anterior). Un semáforo para el tablero de palabras que hace que no puedan escribir 2 threads distintos al mismo tiempo sobre este, osea que no puedan gritar 'palabra' exactamente al mismo tiempo y un tablero de semáforos para cada posicion del tablero de letras. Este último es para que a medida que van formando las palabras no puedan escribir una letra sobre el mismo casillero al mismo tiempo. \\
Finalmente agregamos estos semáforos a la hora de escribir/leer alguno de los dos tableros. Esto podemos observarlo mejor en los siguientes pseudocodigos(vamos a poner el pseudocódigo de lo que agregamos, no de lo que ya estaba):\\
\\


MAIN \\
...
Inicializamos los semaforos\\
...
Cuando llega una nueva conexion \\
Si se establece exitosamente\\
	Creamos un thread\\
	hacemos que este nuevo cliente sea atendido por el thread a traves de la funcion wrapper\\


\\

wrapper\\
Casteo el parametro de entrada a int y se lo envio a atendedor_de_jugador


atendedor_de_jugador\\ \\
...
Si estoy agregando una letra\\
	pido el recurso para escribir esa casilla
	escribo la letra
	libero el recurso de la casilla
...
Si quiero agregar una palabra
	pido el recurso para escribir el tablero de palabras
	escribo todas las letras
	libero el recurso para el tablero de palabras


enviar_tablero\\

pido el recurso para leer el tablero de palabras
leo todo el tablero
libero el recurso para el tablero de palabras


quitar_letras\\

para cada casilla del tablero de letras
	pido el recurso para escribir esa casilla
		escribo el valor VACIO
	libero el recurso

es_ficha_valida_en_palabra\\
...
Pido el recurso para leer la casilla
si la casilla esta vacia
	libero el recurso
	devuelvo false
libero el recurso
...
si no es la primera letra de la palabra 
	...
	para cada casilla entre la nueva letra y las ya puestas
		pido recurso para leer esa casilla
		si esta vacia
			libero el recurso
			devuelvo false
		libero el recurso
						
...


